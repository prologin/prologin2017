

%
% This file was generated using gen/make_tex.rtex
% Do not modify unless you are absolutely sure of what you are doing
%



\noindent \begin{tabular}{lp{11cm}}
\textbf{Constante:} & TAILLE\_ETABLI \\
\textbf{Valeur:} & 6 \\
\textbf{Description:} & Taille de l’établi de travail (longueur et largeur) \\
\end{tabular}
\vspace{0.2cm} \\



\noindent \begin{tabular}{lp{11cm}}
\textbf{Constante:} & NB\_TOURS \\
\textbf{Valeur:} & 150 \\
\textbf{Description:} & Nombre de tours à jouer avant la fin de l’affrontement \\
\end{tabular}
\vspace{0.2cm} \\



\noindent \begin{tabular}{lp{11cm}}
\textbf{Constante:} & NB\_TYPE\_CASES \\
\textbf{Valeur:} & 6 \\
\textbf{Description:} & Taille de l’énumération ``case\_type`` \\
\end{tabular}
\vspace{0.2cm} \\





\functitle{case\_type} \\
\noindent
\begin{tabular}[t]{@{\extracolsep{0pt}}>{\bfseries}lp{10cm}}
Description~: & Types de cases \\
Valeurs~: &
\small
\begin{tabular}[t]{@{\extracolsep{0pt}}lp{7cm}}
    
        \textsl{VIDE}~: & Case vide \\
    
        \textsl{PLOMB}~: & Plomb ; transmutable en or \\
    
        \textsl{FER}~: & Fer ; transmutable en or \\
    
        \textsl{CUIVRE}~: & Cuivre ; transmutable en or \\
    
        \textsl{SOUFRE}~: & Soufre ; transmutable en catalyseur \\
    
        \textsl{MERCURE}~: & Mercure ; transmutable en catalyseur \\
    
\end{tabular} \\
\end{tabular}



\functitle{element\_propriete} \\
\noindent
\begin{tabular}[t]{@{\extracolsep{0pt}}>{\bfseries}lp{10cm}}
Description~: & Types de propriétés des éléments \\
Valeurs~: &
\small
\begin{tabular}[t]{@{\extracolsep{0pt}}lp{7cm}}
    
        \textsl{AUCUNE}~: & Les cases vides ne contiennent pas d’élément, et n’ont donc aucune propriété \\
    
        \textsl{TRANSMUTABLE\_OR}~: & Élement transmutable en or \\
    
        \textsl{TRANSMUTABLE\_CATALYSEUR}~: & Élément transmutable en catalyseur \\
    
\end{tabular} \\
\end{tabular}



\functitle{erreur} \\
\noindent
\begin{tabular}[t]{@{\extracolsep{0pt}}>{\bfseries}lp{10cm}}
Description~: & Erreurs possibles \\
Valeurs~: &
\small
\begin{tabular}[t]{@{\extracolsep{0pt}}lp{7cm}}
    
        \textsl{OK}~: & L’action a été exécutée avec succès \\
    
        \textsl{POSITION\_INVALIDE}~: & La position spécifiée n’est pas sur l’établi \\
    
        \textsl{PLACEMENT\_INVALIDE}~: & Les deux positions ne correspondent pas à des cases adjacentes \\
    
        \textsl{PLACEMENT\_IMPOSSIBLE}~: & Les cases ciblées ne sont pas vides \\
    
        \textsl{PLACEMENT\_INCORRECT}~: & Un des deux éléments de l'échantillon doit être placé adjacent à un élément du même type déjà présent sur l'établi \\
    
        \textsl{CASE\_VIDE}~: & La case ciblée est vide \\
    
        \textsl{ECHANTILLON\_INCOMPLET}~: & L’échantillon doit contenir deux éléments. \\
    
        \textsl{ECHANTILLON\_INVALIDE}~: & L’échantillon doit contenir au moins un des éléments de l’échantillon reçu auparavant \\
    
        \textsl{AUCUN\_CATALYSEUR}~: & Aucun catalyseur disponible \\
    
        \textsl{CATALYSE\_INVALIDE}~: & L'élément de destination ne peut pas être vide. \\
    
        \textsl{DEJA\_POSE}~: & L’échantillon a déjà été posé ce tour-ci \\
    
        \textsl{DEJA\_DONNE}~: & L’échantillon a déjà été donné ce tour-ci \\
    
\end{tabular} \\
\end{tabular}



\functitle{action\_type} \\
\noindent
\begin{tabular}[t]{@{\extracolsep{0pt}}>{\bfseries}lp{10cm}}
Description~: & Types d’actions \\
Valeurs~: &
\small
\begin{tabular}[t]{@{\extracolsep{0pt}}lp{7cm}}
    
        \textsl{ACTION\_PLACER}~: & Action ``placer\_echantillon`` \\
    
        \textsl{ACTION\_TRANSMUTER}~: & Action ``transmuter`` \\
    
        \textsl{ACTION\_CATALYSER}~: & Action ``catalyser`` \\
    
        \textsl{ACTION\_DONNER\_ECHANTILLON}~: & Action ``donner\_echantillon`` \\
    
\end{tabular} \\
\end{tabular}





\functitle{position}

\begin{lst-c++}
struct position {
    int ligne;
    int colonne;
};
\end{lst-c++}

\noindent
\begin{tabular}[t]{@{\extracolsep{0pt}}>{\bfseries}lp{10cm}}
Description~: & Position sur la carte, donnée par deux coordonnées \\
Champs~: &
\small
\begin{tabular}[t]{@{\extracolsep{0pt}}lp{7cm}}
    
        \textsl{ligne}~: & Coordonnée : ligne \\
    
        \textsl{colonne}~: & Coordonnée : colonne \\
    
\end{tabular} \\
\end{tabular}



\functitle{echantillon}

\begin{lst-c++}
struct echantillon {
    case_type element1;
    case_type element2;
};
\end{lst-c++}

\noindent
\begin{tabular}[t]{@{\extracolsep{0pt}}>{\bfseries}lp{10cm}}
Description~: & Échantillon, défini par deux types d’éléments \\
Champs~: &
\small
\begin{tabular}[t]{@{\extracolsep{0pt}}lp{7cm}}
    
        \textsl{element1}~: & Élément 1 \\
    
        \textsl{element2}~: & Élément 2 \\
    
\end{tabular} \\
\end{tabular}



\functitle{position\_echantillon}

\begin{lst-c++}
struct position_echantillon {
    position pos1;
    position pos2;
};
\end{lst-c++}

\noindent
\begin{tabular}[t]{@{\extracolsep{0pt}}>{\bfseries}lp{10cm}}
Description~: & Position d’un échantillon, donnée par deux positions adjacentes \\
Champs~: &
\small
\begin{tabular}[t]{@{\extracolsep{0pt}}lp{7cm}}
    
        \textsl{pos1}~: & Position de l’élément 1 de l’échantillon \\
    
        \textsl{pos2}~: & Position de l’élément 2 de l’échantillon \\
    
\end{tabular} \\
\end{tabular}



\functitle{action\_hist}

\begin{lst-c++}
struct action_hist {
    action_type atype;
    position pos1;
    position pos2;
    int id_apprenti;
    case_type nouvelle_case;
};
\end{lst-c++}

\noindent
\begin{tabular}[t]{@{\extracolsep{0pt}}>{\bfseries}lp{10cm}}
Description~: & Action représentée dans l’historique. L’action ``placer\_echantillon`` utilise ``pos1`` et ``pos2``. L’action ``transmuter`` utilise ``pos1``. L’action ``catalyser`` utilise ``pos1``, ``id\_apprenti`` et ``nouvelle\_case``. L’action ``donner\_echantillon`` n’est pas représentée dans l’historique, car ``echantillon\_tour`` donne l’information. \\
Champs~: &
\small
\begin{tabular}[t]{@{\extracolsep{0pt}}lp{7cm}}
    
        \textsl{atype}~: & Type de l’action \\
    
        \textsl{pos1}~: & Position, pour les actions placer (1er élément), transmuter et catalyser \\
    
        \textsl{pos2}~: & Position, pour l’action placer (2e élément) \\
    
        \textsl{id\_apprenti}~: & ID de l’apprenti, pour l’action catalyser \\
    
        \textsl{nouvelle\_case}~: & Élément pour l’action catalyser \\
    
\end{tabular} \\
\end{tabular}




\begin{minipage}{\linewidth}
\functitle{placer\_echantillon}

\begin{lst-c++}
erreur placer_echantillon(position pos1, position pos2)
\end{lst-c++}

\noindent
\begin{tabular}[t]{@{\extracolsep{0pt}}>{\bfseries}lp{10cm}}
Description~: & Place l’échantillon du tour sur l’établi, avec les coordonnées de deux cases adjacentes. \\


Parametres~: &
\begin{tabular}[t]{@{\extracolsep{0pt}}ll}
    
    
      
        \textsl{pos1}~: & Case du terrain où doit être posé le premier élément de l’échantillon \\
      
    
      
        \textsl{pos2}~: & Case du terrain où doit être posé le second élément de l’échantillon \\
      
    
  \end{tabular} \\






\end{tabular} \\[0.3cm]
\end{minipage}


\begin{minipage}{\linewidth}
\functitle{transmuter}

\begin{lst-c++}
erreur transmuter(position pos)
\end{lst-c++}

\noindent
\begin{tabular}[t]{@{\extracolsep{0pt}}>{\bfseries}lp{10cm}}
Description~: & Provoque la transformation chimique de l’élément à la case ciblée, ainsi que tous les éléments adjacents du même type, ceux du même type adjacents à ces derniers, etc. Ils disparaissent alors tous dans leur transmutation en or ou en catalyseur. \\


Parametres~: &
\begin{tabular}[t]{@{\extracolsep{0pt}}ll}
    
    
      
        \textsl{pos}~: & Case de l’établi dont la région doit être activée \\
      
    
  \end{tabular} \\






\end{tabular} \\[0.3cm]
\end{minipage}


\begin{minipage}{\linewidth}
\functitle{catalyser}

\begin{lst-c++}
erreur catalyser(position pos, int id_apprenti, case_type terrain)
\end{lst-c++}

\noindent
\begin{tabular}[t]{@{\extracolsep{0pt}}>{\bfseries}lp{10cm}}
Description~: & Utilise un catalyseur sur la case ciblée de l'apprenti indiqué. Transforme l’ancien élément en l’élément indiqué. \\


Parametres~: &
\begin{tabular}[t]{@{\extracolsep{0pt}}ll}
    
    
      
        \textsl{pos}~: & Case de l’élément qui doit être transmuté \\
      
    
      
        \textsl{id\_apprenti}~: & Identifiant de l’apprenti dont l’élément est ciblé \\
      
    
      
        \textsl{terrain}~: & Type d’élément qui doit remplacer l’ancien \\
      
    
  \end{tabular} \\






\end{tabular} \\[0.3cm]
\end{minipage}


\begin{minipage}{\linewidth}
\functitle{donner\_echantillon}

\begin{lst-c++}
erreur donner_echantillon(echantillon echantillon_donne)
\end{lst-c++}

\noindent
\begin{tabular}[t]{@{\extracolsep{0pt}}>{\bfseries}lp{10cm}}
Description~: & Définit l’échantillon que l’adversaire recevra à son prochain tour. \\


Parametres~: &
\begin{tabular}[t]{@{\extracolsep{0pt}}ll}
    
    
      
        \textsl{echantillon\_donne}~: & Échantillon que l’adversaire recevra à son prochain tour \\
      
    
  \end{tabular} \\






\end{tabular} \\[0.3cm]
\end{minipage}


\begin{minipage}{\linewidth}
\functitle{type\_case}

\begin{lst-c++}
case_type type_case(position pos, int id_apprenti)
\end{lst-c++}

\noindent
\begin{tabular}[t]{@{\extracolsep{0pt}}>{\bfseries}lp{10cm}}
Description~: & Renvoie le type d’une case donnée, ou 0 si la case est invaide. \\


Parametres~: &
\begin{tabular}[t]{@{\extracolsep{0pt}}ll}
    
    
      
        \textsl{pos}~: & Case choisie \\
      
    
      
        \textsl{id\_apprenti}~: & Apprenti choisi \\
      
    
  \end{tabular} \\






\end{tabular} \\[0.3cm]
\end{minipage}


\begin{minipage}{\linewidth}
\functitle{est\_vide}

\begin{lst-c++}
bool est_vide(position pos, int id_apprenti)
\end{lst-c++}

\noindent
\begin{tabular}[t]{@{\extracolsep{0pt}}>{\bfseries}lp{10cm}}
Description~: & Indique si une case donnée est vide ou contient un élément. Renvoie faux en cas d'erreur. \\


Parametres~: &
\begin{tabular}[t]{@{\extracolsep{0pt}}ll}
    
    
      
        \textsl{pos}~: & Case choisie \\
      
    
      
        \textsl{id\_apprenti}~: & Apprenti choisi \\
      
    
  \end{tabular} \\






\end{tabular} \\[0.3cm]
\end{minipage}


\begin{minipage}{\linewidth}
\functitle{propriete\_case}

\begin{lst-c++}
element_propriete propriete_case(position pos, int id_apprenti)
\end{lst-c++}

\noindent
\begin{tabular}[t]{@{\extracolsep{0pt}}>{\bfseries}lp{10cm}}
Description~: & Renvoie la propriété de l’élément sur une case donnée. Un élément invalide n'a pas de propriété. \\


Parametres~: &
\begin{tabular}[t]{@{\extracolsep{0pt}}ll}
    
    
      
        \textsl{pos}~: & Case choisie \\
      
    
      
        \textsl{id\_apprenti}~: & Apprenti choisi \\
      
    
  \end{tabular} \\






\end{tabular} \\[0.3cm]
\end{minipage}


\begin{minipage}{\linewidth}
\functitle{propriete\_case\_type}

\begin{lst-c++}
element_propriete propriete_case_type(case_type ctype)
\end{lst-c++}

\noindent
\begin{tabular}[t]{@{\extracolsep{0pt}}>{\bfseries}lp{10cm}}
Description~: & Renvoie la propriété d’un type de case donné. \\


Parametres~: &
\begin{tabular}[t]{@{\extracolsep{0pt}}ll}
    
    
      
        \textsl{ctype}~: & Type de case \\
      
    
  \end{tabular} \\






\end{tabular} \\[0.3cm]
\end{minipage}


\begin{minipage}{\linewidth}
\functitle{taille\_region}

\begin{lst-c++}
int taille_region(position pos, int id_apprenti)
\end{lst-c++}

\noindent
\begin{tabular}[t]{@{\extracolsep{0pt}}>{\bfseries}lp{10cm}}
Description~: & Renvoie la taille de la région à laquelle appartient un élément. Renvoie -1 si la position est invalide. \\


Parametres~: &
\begin{tabular}[t]{@{\extracolsep{0pt}}ll}
    
    
      
        \textsl{pos}~: & Case choisie \\
      
    
      
        \textsl{id\_apprenti}~: & Apprenti choisi \\
      
    
  \end{tabular} \\






\end{tabular} \\[0.3cm]
\end{minipage}


\begin{minipage}{\linewidth}
\functitle{positions\_region}

\begin{lst-c++}
position array positions_region(position pos, int id_apprenti)
\end{lst-c++}

\noindent
\begin{tabular}[t]{@{\extracolsep{0pt}}>{\bfseries}lp{10cm}}
Description~: & Renvoie la liste des positions des cases composant la région à laquelle appartient un élément donné. Renvoie une liste vide en cas d'erreur. \\


Parametres~: &
\begin{tabular}[t]{@{\extracolsep{0pt}}ll}
    
    
      
        \textsl{pos}~: & Case choisie \\
      
    
      
        \textsl{id\_apprenti}~: & Apprenti choisi \\
      
    
  \end{tabular} \\






\end{tabular} \\[0.3cm]
\end{minipage}


\begin{minipage}{\linewidth}
\functitle{placement\_possible\_echantillon}

\begin{lst-c++}
bool placement_possible_echantillon(echantillon echantillon_a_placer,
                                        position pos1, position pos2, int id_apprenti)
\end{lst-c++}

\noindent
\begin{tabular}[t]{@{\extracolsep{0pt}}>{\bfseries}lp{10cm}}
Description~: & Détermine si le placement d’un échantillon est valide. \\


Parametres~: &
\begin{tabular}[t]{@{\extracolsep{0pt}}ll}
    
    
      
        \textsl{echantillon\_a\_placer}~: & Échantillon à placer \\
      
    
      
        \textsl{pos1}~: & Case du terrain où doit être posé le premier élément\\ &de l’échantillon \\
      
    
      
        \textsl{pos2}~: & Case du terrain où doit être posé le second élément de\\ &l’échantillon \\
      
    
      
        \textsl{id\_apprenti}~: & Apprenti possédant l’établi où poser l’échantillon \\
      
    
  \end{tabular} \\






\end{tabular} \\[0.3cm]
\end{minipage}


\begin{minipage}{\linewidth}
\functitle{placements\_possible\_echantillon}

\begin{lst-c++}
position_echantillon array placements_possible_echantillon(
        echantillon echantillon_a_placer, int id_apprenti)
\end{lst-c++}

\noindent
\begin{tabular}[t]{@{\extracolsep{0pt}}>{\bfseries}lp{10cm}}
Description~: & Renvoie la liste des placements possibles pour un échantillon donné sur l’établi d’un apprenti donné. Renvoie une liste vide en cas d'erreur. \\


Parametres~: &
\begin{tabular}[t]{@{\extracolsep{0pt}}ll}
    
    
      
        \textsl{echantillon\_a\_placer}~: & Échantillon à placer \\
      
    
      
        \textsl{id\_apprenti}~: & Apprenti possédant l’établi où poser l’échantillon \\
      
    
  \end{tabular} \\






\end{tabular} \\[0.3cm]
\end{minipage}


\begin{minipage}{\linewidth}
\functitle{historique}

\begin{lst-c++}
action_hist array historique()
\end{lst-c++}

\noindent
\begin{tabular}[t]{@{\extracolsep{0pt}}>{\bfseries}lp{10cm}}
Description~: & Renvoie la liste des actions jouées par l’adversaire pendant son tour, dans l’ordre chronologique. \\







\end{tabular} \\[0.3cm]
\end{minipage}


\begin{minipage}{\linewidth}
\functitle{moi}

\begin{lst-c++}
int moi()
\end{lst-c++}

\noindent
\begin{tabular}[t]{@{\extracolsep{0pt}}>{\bfseries}lp{10cm}}
Description~: & Renvoie votre numéro d’apprenti. \\







\end{tabular} \\[0.3cm]
\end{minipage}


\begin{minipage}{\linewidth}
\functitle{adversaire}

\begin{lst-c++}
int adversaire()
\end{lst-c++}

\noindent
\begin{tabular}[t]{@{\extracolsep{0pt}}>{\bfseries}lp{10cm}}
Description~: & Renvoie le numéro d’apprenti de votre adversaire. \\







\end{tabular} \\[0.3cm]
\end{minipage}


\begin{minipage}{\linewidth}
\functitle{score}

\begin{lst-c++}
int score(int id_apprenti)
\end{lst-c++}

\noindent
\begin{tabular}[t]{@{\extracolsep{0pt}}>{\bfseries}lp{10cm}}
Description~: & Renvoie la quantité d’or amassée par l’apprenti désigné par le numéro ``id\_apprenti``. Renvoie 0 si ``id\_apprenti`` est invalide (attention, le score d’un apprenti valide peut aussi être 0). \\


Parametres~: &
\begin{tabular}[t]{@{\extracolsep{0pt}}ll}
    
    
      
        \textsl{id\_apprenti}~: & Identifiant de l’apprenti \\
      
    
  \end{tabular} \\






\end{tabular} \\[0.3cm]
\end{minipage}


\begin{minipage}{\linewidth}
\functitle{tour\_actuel}

\begin{lst-c++}
int tour_actuel()
\end{lst-c++}

\noindent
\begin{tabular}[t]{@{\extracolsep{0pt}}>{\bfseries}lp{10cm}}
Description~: & Renvoie le numéro du tour actuel. \\







\end{tabular} \\[0.3cm]
\end{minipage}


\begin{minipage}{\linewidth}
\functitle{annuler}

\begin{lst-c++}
bool annuler()
\end{lst-c++}

\noindent
\begin{tabular}[t]{@{\extracolsep{0pt}}>{\bfseries}lp{10cm}}
Description~: & Annule la dernière action. Renvoie ``false`` quand il n’y a pas d’action à annuler ce tour-ci. \\







\end{tabular} \\[0.3cm]
\end{minipage}


\begin{minipage}{\linewidth}
\functitle{nombre\_catalyseurs}

\begin{lst-c++}
int nombre_catalyseurs()
\end{lst-c++}

\noindent
\begin{tabular}[t]{@{\extracolsep{0pt}}>{\bfseries}lp{10cm}}
Description~: & Indique le nombre de catalyseurs en votre possession. \\







\end{tabular} \\[0.3cm]
\end{minipage}


\begin{minipage}{\linewidth}
\functitle{echantillon\_tour}

\begin{lst-c++}
echantillon echantillon_tour()
\end{lst-c++}

\noindent
\begin{tabular}[t]{@{\extracolsep{0pt}}>{\bfseries}lp{10cm}}
Description~: & Indique l’échantillon reçu pour ce tour. \\







\end{tabular} \\[0.3cm]
\end{minipage}


\begin{minipage}{\linewidth}
\functitle{a\_pose\_echantillon}

\begin{lst-c++}
bool a_pose_echantillon()
\end{lst-c++}

\noindent
\begin{tabular}[t]{@{\extracolsep{0pt}}>{\bfseries}lp{10cm}}
Description~: & Indique si l’échantillon reçu pour ce tour a déjà été posé. \\







\end{tabular} \\[0.3cm]
\end{minipage}


\begin{minipage}{\linewidth}
\functitle{a\_donne\_echantillon}

\begin{lst-c++}
bool a_donne_echantillon()
\end{lst-c++}

\noindent
\begin{tabular}[t]{@{\extracolsep{0pt}}>{\bfseries}lp{10cm}}
Description~: & Indique si un échantillon a déjà été donné ce tour. \\







\end{tabular} \\[0.3cm]
\end{minipage}


\begin{minipage}{\linewidth}
\functitle{quantite\_transmutation\_or}

\begin{lst-c++}
int quantite_transmutation_or(int taille_region)
\end{lst-c++}

\noindent
\begin{tabular}[t]{@{\extracolsep{0pt}}>{\bfseries}lp{10cm}}
Description~: & Renvoie la quantité d’or (et donc le score) obtenue par la transmutation de ``taille\_region`` éléments transmutables en or. \\


Parametres~: &
\begin{tabular}[t]{@{\extracolsep{0pt}}ll}
    
    
      
        \textsl{taille\_region}~: & Nombre d’éléments d’une région à transmuter \\
      
    
  \end{tabular} \\






\end{tabular} \\[0.3cm]
\end{minipage}


\begin{minipage}{\linewidth}
\functitle{quantite\_transmutation\_catalyseur}

\begin{lst-c++}
int quantite_transmutation_catalyseur(int taille_region)
\end{lst-c++}

\noindent
\begin{tabular}[t]{@{\extracolsep{0pt}}>{\bfseries}lp{10cm}}
Description~: & Renvoie la quantité de catalyseurs obtenue par la transmutation de ``taille\_region`` éléments transmutables en catalyseur. \\


Parametres~: &
\begin{tabular}[t]{@{\extracolsep{0pt}}ll}
    
    
      
        \textsl{taille\_region}~: & Nombre d’éléments d’une région à transmuter \\
      
    
  \end{tabular} \\






\end{tabular} \\[0.3cm]
\end{minipage}


\begin{minipage}{\linewidth}
\functitle{quantite\_transmutation\_catalyseur\_or}

\begin{lst-c++}
int quantite_transmutation_catalyseur_or(int taille_region)
\end{lst-c++}

\noindent
\begin{tabular}[t]{@{\extracolsep{0pt}}>{\bfseries}lp{10cm}}
Description~: & Renvoie la quantité d’or obtenue par la transmutation de ``taille\_region`` éléments transmutables en catalyseur. \\


Parametres~: &
\begin{tabular}[t]{@{\extracolsep{0pt}}ll}
    
    
      
        \textsl{taille\_region}~: & Nombre d’éléments d’une région à transmuter \\
      
    
  \end{tabular} \\






\end{tabular} \\[0.3cm]
\end{minipage}


\begin{minipage}{\linewidth}
\functitle{echantillon\_defaut\_premier\_tour}

\begin{lst-c++}
echantillon echantillon_defaut_premier_tour()
\end{lst-c++}

\noindent
\begin{tabular}[t]{@{\extracolsep{0pt}}>{\bfseries}lp{10cm}}
Description~: & Indique l’échantillon par défaut lors du premier tour \\







\end{tabular} \\[0.3cm]
\end{minipage}


\begin{minipage}{\linewidth}
\functitle{afficher\_etablis}

\begin{lst-c++}
void afficher_etablis()
\end{lst-c++}

\noindent
\begin{tabular}[t]{@{\extracolsep{0pt}}>{\bfseries}lp{10cm}}
Description~: & Affiche l'état actuel des deux établis dans la console. \\







\end{tabular} \\[0.3cm]
\end{minipage}


