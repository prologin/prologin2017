% !TEX encoding = UTF-8 Unicode
% $Header: /cvsroot/latex-beamer/latex-beamer/solutions/conference-talks/conference-ornate-20min.en.tex,v 1.6 2004/10/07 20:53:08 tantau Exp $

\documentclass{beamer}

\mode<presentation>
{
  \usetheme{Warsaw}
  % or ...

  \setbeamercovered{invisible}
  % or whatever (possibly just delete it)
  
  \setbeamertemplate{navigation symbols}{}
  
  \newcommand*\oldmacro{}%
  \let\oldmacro\insertshorttitle%
  \renewcommand*\insertshorttitle{%
    \oldmacro\hfill%
    \insertframenumber\,/\,\inserttotalframenumber}
}

\usepackage[utf8]{inputenc}
% or whatever

\usepackage{times}
\usepackage{multirow}
\usepackage[T1]{fontenc}
\usepackage[french]{babel}
\usepackage{graphicx}

\usepackage{eso-pic}
\usepackage{color}
\usepackage{tikz}
\usepackage{wasysym}

% Or whatever. Note that the encoding and the font should match. If T1
% does not look nice, try deleting the line with the fontenc.

%\title[Petit guide des bonnes pratiques pour la construction et la maintenance d'$\alpha$-extracteurs cosmiques à vacuité]
{}

\titlegraphic{\raisebox{2em}{}}

\author[Prologin]
{\includegraphics{../prologin2017}}

\date
{}

\begin{document}

\definecolor{vert}{rgb}{0.07 0.54 0.07}


\begin{frame}
        \centering \includegraphics[width=0.9\linewidth]{../prologin2017} \\
        \vspace{1.5cm}
    \textbf{Tabula Prologina}
    \vspace{0.5cm}

    \textit{Felix qui potuit cognoscere algorithmum}
\end{frame}

\begin{frame}
    \frametitle{Qui sommes nous?}
    \begin{itemize}
        \item Alchimistes éminents du XVIII\ieme{} siècle
        \item À la recherche du prochain Grand Maître
        \item Il est présent parmis \textbf{vous}!
    \end{itemize}
\end{frame}

\begin{frame}
    \frametitle{Présentation de l'épreuve d'alchimie}
    \begin{itemize}
        \item Deux apprentis travaillent simultanément sur leurs établis
        \item Chacun tente de transmuter ses éléments en or
        \item Celui qui remporte le plus d'or est le plus Grand des deux
    \end{itemize}
\end{frame}

\begin{frame}
    \frametitle{Votre espace de travail}
    \includegraphics[width=\textwidth]{../img/bench-empty}
\end{frame}

\begin{frame}
    \frametitle{Échantillon}
    \begin{block}{Éléments}
        \begin{tabular}{ccccc}
            \includegraphics[width=1.5cm]{../img/material-lead} &
            \includegraphics[width=1.5cm]{../img/material-iron} &
            \includegraphics[width=1.5cm]{../img/material-copper} &
            \includegraphics[width=1.5cm]{../img/material-mercury} &
            \includegraphics[width=1.5cm]{../img/material-sulfur}\\
            Plomb & Fer & Cuivre & Mercure & Soufre
        \end{tabular}
    \end{block}
    \begin{block}{Échantillon}
        Constitué de deux éléments, identiques ou différents.
    \end{block}
    \begin{block}{Placement d'un échantillon}
        \begin{itemize}
            \item Éléments posés sur des cases adjacentes
            \item Un des deux éléments doit être posé adjacent à un élément du
                même type déjà sur l'établi
            \item Si aucun des deux éléments n'est présent, placement libre
        \end{itemize}
    \end{block}
\end{frame}

\begin{frame}
    \frametitle{Donner un échantillon}
    \begin{block}{Choisi pour l’adversaire}
        \begin{itemize}
            \item On ne se sert pas soi-même dans la réserve!
            \item On apporte un échantillon à son camarade
            \item À son tour, on choisi l'échantillon qu'aura l'autre au sien
        \end{itemize}
    \end{block}
    \begin{block}{Contraintes}
        L'échantillon donné doit avoir au moins un élément en commun avec
        l'échantillon reçu.
    \end{block}
    \begin{block}{Oubli}
        Si vous oubliez de donner un échantillon à votre adversaire, il recevra
        la même chose que ce qu’il vous a donné au tour précédent.
    \end{block}
\end{frame}

\begin{frame}
    \frametitle{Transmutation}
    \begin{block}{Voie sèche}
        Transforme \textbf{cuivre}, \textbf{fer} ou \textbf{plomb}, en or.
        \\
        \begin{tabular}{ccccc}
            \includegraphics[width=1.5cm]{../img/material-lead} &
            \includegraphics[width=1.5cm]{../img/material-iron} &
            \includegraphics[width=1.5cm]{../img/material-copper} &
            $\xrightarrow{\text{transmutation}}$ &
            \includegraphics[width=1.5cm]{../img/material-gold}\\
            Plomb & Fer & Cuivre & & Or
        \end{tabular}
    \end{block}
    \begin{block}{Catalysation}
        Transforme \textbf{soufre} ou \textbf{mercure} en catalyseur
        \\
        \begin{tabular}{cccccc}
            \includegraphics[width=1.5cm]{../img/material-mercury} &
            \includegraphics[width=1.5cm]{../img/material-sulfur} &
            $\xrightarrow{\text{transmutation}}$ &
            \includegraphics[width=1.5cm]{../img/material-catalyst} &
            + &
            \includegraphics[width=0.5cm]{../img/material-gold}\\
            Mercure & Soufre & & Catalyseur & & Or\\
        \end{tabular}
    \end{block}
\end{frame}

\begin{frame}
    \frametitle{Catalyse}
    \vspace{1cm}
    \begin{tabular}{ccccccc}
        \includegraphics[width=0.8cm]{../img/material-lead} &
        \includegraphics[width=0.8cm]{../img/material-iron} &
        \includegraphics[width=0.8cm]{../img/material-copper} &
        \includegraphics[width=0.8cm]{../img/material-mercury} &
        \includegraphics[width=0.8cm]{../img/material-sulfur} &
        $\xrightarrow{\text{Catalyse}}$ &
        \includegraphics[width=0.8cm]{../img/material-other}\\
        Plomb & Fer & Cuivre & Mercure & Soufre &
        \includegraphics[width=0.5cm]{../img/material-catalyst} &
        Élément
    \end{tabular}

    \vspace{1cm}
    \begin{itemize}
        \item Un catalyseur transforme un métal vil en un autre
        \item La réaction peut être déclenchée chez l’adversaire!
    \end{itemize}
    \begin{alertblock}{Attention}
        Un catalyseur est instable et doit être utilisé immédiatement
    \end{alertblock}
\end{frame}

\begin{frame}
    \frametitle{Questions}
    Posez vos questions sur le sujet de la finale !
\end{frame}

\begin{frame}
    \frametitle{Tournois intermédiaires}
    \begin{itemize}
        \item Samedi 15~h~42 (tournoi test)
        \item Samedi 17~h~42
        \item Samedi 23~h~42
        \item Dimanche 5~h~42
        \item Dimanche 11~h~42
        \item Dimanche 17~h~42
        \item Lundi 00~h~42 (tournoi final)
    \end{itemize}
\end{frame}

\begin{frame}
    \frametitle{Conférences}
    \begin{itemize}
        \item \textbf{Lundi 11~h~30} Éva Attal : Les métiers de développeurs en salle de marché
    \end{itemize}
\end{frame}

\begin{frame}
    \frametitle{Fin}
    Bonne finale !
\end{frame}

\end{document}
