\newpage
\noindent
C'est un point assuré plein d'admiration, \\
Que le haut \& et le bas n'est qu'une même chose : \\
Pour faire d'une seule en tout le monde enclose,\\
Des effets merveilleux par adaptation.\\

\noindent
D'un seul en a tout fait la méditation,\\
Et pour parents, matrice, \& nourrice, on lui pose,\\
Phœbus, Diane, l'air, \& la terre, où repose,\\
Cette chose en qui gît toute perfection.\\

\noindent
Si on la mue en terre elle a sa force entière :\\
Séparant par grand art, mais facile manière,\\
Le subtil de l'épais, \& la terre du feu.\\

\noindent
De la terre elle monte au ciel, \& puis en terre,\\
Du Ciel elle descend, recevant peu à peu,\\
Les vertus de tous deux qu'en son ventre elle enserre.\\

\small{-- -- \textsc{Table d'émeraude} - Hermes Trismegiste}

\newpage

\section{\textsc{Préambule}}

\lettrine{L}{a} sagesse que j'ai reccueilli ici pour vous, mes disciples, ne doit jamais,
\emph{jamais}, sortir du cadre des leçons que je vous propose.
Contrairement aux habitudes de mes pairs, j'ai décidé d'expliciter ici mes
connaissances sans abstrus distraits et surtout, sans embrumer mon
propos. Cela fait de cet ouvrage le recueil d'Alchimie le plus dangereux qui
ai jamais été rédigé. Imaginez seulement si le profane mettait la main sur ces
lignes !

J'ai élucidé pour vous de nombreux mystères de ce monde. Je suis parvenu à des
miracles devant lesquels tant de mes confrères ont abandonné. J'ai réussi là où
tant d'autres ont échoué\marginpar{Pour n'en citer qu'un, Nicolas Flamel est un
triste imposteur.}. Ce n'est pas
sans apréhension que je vous transmet aujourd'hui mes déductions, mais je ne puis
plus continuer sans l'assistance précieuse d'un apprenti fiable.

J'ai l'honneur de pouvoir clamer que j'ai véritablement percé le mystère du
\emph{Grand Oeuvre}. Je détient le secret de la panacée, et sa simplicité, son
élégance, m'ont ébloui. Cette parcelle de divinité, je ne puis la partager avec
quiconque. Lorsque l'Homme sera éclairé et la civilisation dépouillée de son
impureté, je m'élèverai pour enfin lui confier \emph{l'immortalité}. Je serai
le Prométhée de l'ère moderne !

C'est pour cela que ces travaux sont si importants, et pour cela que je ne
laisserai personne s'y opposer. Si le secret est jamais éventré, par n'importe
lequel d'entre vous, soyez sûrs que ce ne sera pas oublié ni impuni.

\section{\textsc{Transmutation}}

\lettrine{N}{ul} ignore qu'il faut agir sur la \emph{Materia Prima}, alors un métal vil,
pour le transformer en or, car c'est là notre premier but. C'est cette
opération que nous appelons la \textbf{transmutation}. De nombreux
alchimistes renommés, et de nombreux charlatans également, se sont penchés sur
ce sujet avec l'espoir vain d'y trouver la richesse.

Mais L'Alchimie est un art qui se veut pur, et quiconque tente de produire
des métaux nobles pour son seul profit ne pourra qu'échouer. C'est pour cela,
uniquement, que j'ai réussi là où de nombreux amateurs se sont vu vaincus.

\subsection{\textsc{Méthode}}

\lettrine{V}{otre} Maître préconise une transmutation rapide par \emph{voie sèche} sur
les trois métaux vils suivants : le \textbf{cuivre}, le \textbf{fer}, ou le
\textbf{plomb}. L'ajout de l'alkahest au moment de l'oeuvre au rouge et
l'isolation selon le principe des feuillets paracelsiens entraînera la réaction
métamorphique appropriée\footnote{"\textit{Mais ça ne veut rien dire !"}
s'exclamme votre voisin, perdu.}\marginpar{Insérez ici une référence quelconque}.
Agissez avec la prudence nécessaire — la difficulté de cette méthode réside dans la
minutie dont vous devrez faire preuve.

Il vous faudra une quantité généreuse de \emph{Materia Prima} pour un résultat
satisfaisant. De plus, n'essayez pas d'user de plusieurs métaux ensemble. Leurs
corps étant radicalement différents, leurs esprits ne peuvent se mêler et vous
n'obtiendrez que rien d'autre que des cendres\footnote{Et des brûlures.}.

\subsection{\textsc{De la transmutation en métaux vils}}

\lettrine{L}{a} transmutation de \emph{Materia Prima} en autres métaux vils est plus utile
qu'on ne pourrait le croire, notamment lorsque la transmutation en or n'est pas
directement possible. J'appelle cette action la \emph{catalysation}, pour des
raisons évidentes. À l'aide de \textbf{soufre} ou de \textbf{mercure} —
éléments qui se doivent d'être familiers à tout alchimiste qui se respecte — il
est possible de transmuter n'importe quel métal vil en un autre métal vil grâce
à la sublimation des éléments sus-cités.

% Insérez ici un schéma random

Usez de ce savoir avec parcimonie et sagesse.

\section{}


