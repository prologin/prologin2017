\newpage

\thispagestyle{empty}

\newlength{\savedparindent}
\setlength{\savedparindent}{\parindent}

\setlength{\parindent}{-4em}

% halfr: testée mais pas convaincu par le rendu
% \setromanfont{Primitive}
% \setromanfont{ArtNoveauDecadente}

\fontsize{25pt}{25pt}\selectfont{}
\textsc{Nouvelles découvertes en Alchimie \& principes de manipulation de la
table alchimique double par Yosef Mercatus.}

\vspace*{1cm}

\fontsize{20pt}{20pt}\selectfont{}
Très grand \& très excellent Philosophe \& célèbre Mathématicien, Prince des
sectateurs Hermétiques \& Algorithmiques.

\vspace*{1cm}

\textit{Textes tirés de nombreux ouvrages du grand père de Yosef Mercatus.
Augmenté des notes manuscrites trouvées dans ceux-ci.  Extraits de lettres
reçues et envoyées par celui-là même, collectées par son aïeul.  Nombre de
textes originaux traduits du latin.  }

\textit{Auxquels on a ajouté en Appendice les Opinions certaines à l'attention
de l'apprenti alchimiste et de son compilateur, personnes très-doctes.}

\setlength{\parindent}{\savedparindent}

\begin{flushright}
À Paris
\end{flushright}

\newpage

\normalsize

\noindent{}C'est un point assuré plein d'admiration,\\
Que le haut \& et le bas n'est qu'une même chose :\\
Pour faire d'une seule en tout le monde enclose,\\
Des effets merveilleux par adaptation.\\
\\
D'un seul en a tout fait la méditation,\\
Et pour parents, matrice, \& nourrice, on lui pose,\\
Phœbus, Diane, l'air, \& la terre, où repose,\\
Cette chose en qui gît toute perfection.\\
\\
Si on la mue en terre elle a sa force entière :\\
Séparant par grand art, mais facile manière,\\
Le subtil de l'épais, \& la terre du feu.\\
\\
De la terre elle monte au ciel, \& puis en terre,\\
Du Ciel elle descend, recevant peu à peu,\\
Les vertus de tous deux qu'en son ventre elle enserre.\\

\textsc{Table d'émeraude} - Hermes Trismegiste

\newpage

\section{\textsc{Préambule}}

\lettrine{L}{e} savoir que j'ai recueilli ici pour vous, mes apprentis, ne doit jamais,
\emph{jamais}, sortir du cadre de l'enseignement que je vous propose.
Contrairement aux habitudes de mes confrères, j'ai décidé d'expliciter ici mes
connaissances sans fioritures abstraites et surtout, sans obscurcir mon
discours. Ceci fait de ce manuscrit le recueil d'Alchimie le plus dangereux qui
ai jamais été écrit. Imaginez seulement si le profane mettait le main sur ces
lignes !

J'ai élucidé pour vous de nombreux mystères de ce monde. Je suis parvenu à des
miracles devant lesquels tant de mes confrères ont abandonné. J'ai réussi là où
tant d'autres ont échoué \footnote{Pour exemple, Nicolas Flamel}. Ce n'est pas
sans appréhension que je vous transmets aujourd'hui mes conclusions, mais je ne puis
plus continuer sans l'aide précieuse d'un apprenti fiable.

Effectivement, j'ai la fierté de pouvoir annoncer que j'ai percé le mystère du
\emph{Grand Œuvre}. Je détiens le secret de la panacée, et sa simplicité, son
élégance, m'ont ébloui. Cette parcelle de divinité, je ne puis la partager avec
n'importe qui. Lorsque l'homme sera sage et la civilisation dépouillée de son
impureté, je ressurgirai pour enfin lui donner \emph{l'immortalité}. Je serai
le Prométhée de l'ère moderne !

Sachez-le : j'ai assuré mes arrières. Ne tentez pas de me tromper, ne tentez
pas de me voler, car je le saurai, et vous regretterez amèrement toute
incartade.

Je vous regarde à tout instant.

\section{\textsc{Transmutation}}

\lettrine{N}{ul} ignore qu'il faut agir sur la \emph{Materia Prima}, alors un métal vil,
pour le transformer en or, car c'est là notre premier but. De nombreux
alchimistes renommés, et de nombreux charlatans également, se sont penchés sur
ce sujet avec l'espoir vain d'y trouver la richesse.

Mais L'Alchimie est un art qui se veut pur, et quiconque tente de produire
des métaux nobles pour son seul profit ne pourra qu'échouer. C'est pour cela,
uniquement, que j'ai réussi là où de nombreux amateurs se sont vu vaincus.

\subsection{\textsc{Méthode}}

\lettrine{V}{otre} Maître préconise une transmutation rapide par \emph{voie sèche} sur
les trois métaux vils suivants : le \textbf{cuivre}, le \textbf{fer}, ou le
\textbf{plomb}. L'ajout de l'alkahest au moment de l'œuvre au rouge et
l'isolation selon le principe des feuillets paracelsiens entraînera la réaction
métamorphique appropriée. Soyez prudent — la difficulté de cette méthode réside dans la
minutie donc vous devrez faire preuve. Si vous vous montrez trop pressé, toute
la préparation risque d'exploser !

Il vous faudra une quantité généreuse de \emph{Materia Prima} pour un résultat
satisfaisant. De plus, n'essayez pas d'user de plusieurs métaux ensemble. Leurs
corps étant radicalement différents, leurs esprits ne peuvent se mêler et vous
n'obtiendrez que cendres et déceptions.

\subsection{\textsc{De la transmutation en métaux vils}}

\lettrine{L}{a} transmutation de \emph{Materia Prima} en autres métaux nobles
est plus utile qu'on ne pourrait le croire, notamment lorsque la transmutation
en or n'est pas directement possible. J'appelle cette action la
\emph{catalysation}, pour des raisons évidentes. À l'aide de \textbf{soufre} ou
de \textbf{mercure} — éléments qui se doivent d'être familiers à tout alchimiste
qui se respecte — il est possible de transmuter n'importe quel métal vil en un
autre métal vil grâce à la sublimation de ces éléments sus-cités.

% Insérez ici un schéma random

Usez de ce savoir avec parcimonie et sagesse.

\section{}


